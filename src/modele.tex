%\documentclass[overfullbox,english,blackandwhite,openan0,hideweeklyreports,expandedcontributors,noseparatecontributorspage,nodate]{polytech}
\documentclass[overfullbox]{polytech}

\usepackage{lipsum}

\schooldepartment{di}
\typereport{prd}
\reportyear{2015-2016}

\title{Le titre principal}
\subtitle{Un eventuel sous titre}
\reportlogo{polytech/polytech}



\student[DI5]{Pierre}{Premier}{pierre.premier@univ-tours.fr}
\student[DI4]{Paul}{Deuxième}{paul2@univ-tours.fr}
\student{Jacques}{Un peu plus Long}{j.unpeupluslong@example.com}
\academicsupervisor[Département infomatique]{Sébastien}{Aupetit}{aupetit@univ-tours.fr}
\academicsupervisor{Machin}{Chose}{machin.chose@univ-tours.fr}
\companysupervisor[Fonction très importante]{Un}{Gars}{un.gars@example.com}
\companysupervisor{Un deuxième}{Gars}{undeuxieme.gars@example.com}
\company[polytech/polytech]{Laboratoire Informatique}{64 avenue Jean Portalis, 37200 Tours}{li.univ-tours.fr}
%\company{Laboratoire Informatique}{64 avenue Jean Portalis, 37200 Tours}{li.univ-tours.fr}

\motcle{mot}
\motcle{clé}
\motcle{deux mots}
\resume{\lipsum[1]}
\abstract{\lipsum[1]}

\keyword{mot}
\keyword{clé}
\keyword{deux mots}

\begin{document}

\maketitle

\unnumberedchapter[intro]{Introduction}

\today


\lipsum[1-10]

\chapter{Avec numéro}

\part{essai}
\label{part:essai}

\chapter{truc}
\lipsum[1-20]

\partref{part:essai}

\section{Ma section}
\lipsum[1-5]

\section{Ma section}
\lipsum[1-5]

\subsection{Ma section}
\lipsum[1-5]
\subsection{Ma section}
\lipsum[1-5]
\subsubsection{Ma section}
\lipsum[1-5]
\subsubsection{Ma section}
\lipsum[1-5]
\paragraph{Ma section}
\lipsum[1-5]
\paragraph{Ma section}
\lipsum[1-5]
\subparagraph{Ma section}
\lipsum[1-5]
\subparagraph{Ma section}
\lipsum[1-5]

\unnumberedchapter{Conclusion}


\makeatletter

% lost in translations :)
\def\polytech@weeklyreports@title@en{Weekly reports}
\def\polytech@weeklyreports@title@fr{Comptes rendus hebdomadaires}
\newcommand{\polytech@weeklyreports@reporttitle@en}[2]{Report ###1 (#2)}
\newcommand{\polytech@weeklyreports@reporttitle@fr}[2]{Compte rendu n°#1\xspace du #2}


% the weekly report counter
\newcounter{polytech@weeklyreport@counter}

\notbool{polytech@hideweeklyreports}{%
	% weekly reports enabled
	\newenvironment{weeklyreports}{%
		\chapter*{\csuse{polytech@weeklyreports@title@\csuse{polytech@lang}}}
		\begin{description}
	}{%
		\end{description}
	}
	\newcommand{\weeklyreport}[1]{%
		\refstepcounter{polytech@weeklyreport@counter}%
		\item[\textbf{\csuse{polytech@weeklyreports@reporttitle\csuse{polytech@lang}}{\thepolytech@weeklyreport@counter}{#1}}]\mbox{ }\\%
	}
}{%
	% weekly reports are disabled
	\excludecomment{weeklyreports}%
}

\makeatother





\begin{weeklyreports}
	\weeklyreport{XX/XX/XXXX}
		\lipsum[1-2]
	\weeklyreport{XX/XX/YYYY}
		\lipsum[3]
\end{weeklyreports}


%--------------------------------------------------------------------------------
%si on donne des annexes :
\appendix
\addcontentsline{toc}{part}{\numberline{}Annexes}

\cite{key1,key2}
\nocite{*}

\bibliographystyle{polytech}
\bibliography{modele}







\makelastpage



\end{document}


