\documentclass[overfullbox,hideweeklyreports,noseparatecontributorspage,nodate]{polytech/polytech}

\schooldepartment{di}
\typereport{none}
\reportyear{Version du \today}
\title{Mode d'emploi de la classe de document \texttt{polytech}}
\subtitle{Descriptif des commandes}
\reportlogo{polytech/modeemploi}

\makeatother
\csdef{polytech@output@contributors}{
\textbf{Auteur(s):}
\href{mailto:aupetit@univ-tours.fr}{Sébastien Aupetit}\par
}
\csdef{polytech@output@contributorspage}{}
\csdef{polytech@output@students@simple}{}
\makeatletter


% On indique les mots clés avec \motcle{mot clé} en français et \keyword{keyword} en anglais
% Le résumé significatif et descriptif du contenu du rapport en 5 à 10 lignes se spécifie par \resume{...} en français et \abstract{...} en anglais
% Attention : tout doit tenir sur la dernière page
\resume{Ce document est le mode d'emploi des commandes principales de la classe de document \texttt{polytech}.}
% chaque mot clé ou groupe de mots clés est défini via la commande \motcle en français
\motcle{Mode d'emploi}

% résumé en anglais
\abstract{This document is a how-to on commandes of the \texttt{polytech} document class.}

% chaque mot clé ou groupe de mots clés est défini via la commande \motcle en anglais
% Attention : il n'y a pas forcément une traduction directe entre les mots clés et les keywords
\keyword{How-to}

\begin{document}
\maketitle

\chapter{La classe de document}

\begin{center}
\textbf{La classe de document a été prévue pour être utilisée avec \latexcode{pdflatex} au lieu de \latexcode{latex}. Prenez donc soin de configurer correctement votre éditeur.}
\end{center}

La classe de document a été prévue de manière à permettre la génération de plusieurs documents dérivés à partir de vos sources LaTeX. Les options de \texttt{\textbackslash{}documentclass} sont donnés par la \autoref{tab:documentclassoptions}. La plupart des options sont réservés pour un usage ultérieur. Lors de la rédaction de votre rapport, il vous suffit de débuter votre document avec l'une des deux commandes suivantes. L'utilisation de l'option \latexcode{overfullbox} est recommander pour vous aider à identifier les débordements de texte.
\begin{latexsource}
\documentclass{polytech}
\end{latexsource}
ou
\begin{latexsource}
\documentclass[overfullbox]{polytech}
\end{latexsource}

\begin{Table}{tab:documentclassoptions}{Liste des options de la classe de document}
\begin{tabu}{|c|X[m]|}
\hline
english&Indique que le document est anglais\\\hline
overfullbox&Provoque l'affichage des débordements (overfull box)\\\hline
blackandwhite&Préparer une version pour impression (sans couleur, sans lien hypertexte, active l'option expandedcontributors)\\\hline
openany&Permet de commencer un chapitre au verso d'une page\\\hline
hideweeklyreports&Desactive l'affichage des comptes rendus hebdomadaires\\\hline
expandedcontributors&Les détails des contributeurs sont inclus\\\hline
noseparatecontributorspage&Desactive la génération de la page listant les contributeurs\\\hline
nodate&Desactive l'affichage de la date de compilation\\\hline
nobibannotations&Desactive l'affichage des annotations dans la bibliographie\\\hline
twoside&Mode recto verso\\\hline
roundtofour&Arrondi le nombre de page du document à un multiple de 4 en mode recto verso\\\hline
\end{tabu}
\end{Table}

\chapter{Première et quatrième de couverture}

\label{chap:premierequatrieme}

La première et quatrième de couverture sont générées automatiquement à partir des informations que vous fournissez. Pour cela, plusieurs commandes à utiliser \textbf{dans le préambule} ont été définies.

\section{Le département de l'école}

Le département de l'école dont dépend le projet est spécifié grâce à la commande \linebreak\latexcode{\\schooldepartment\{dep\}}\footnote{Remarques: le texte déborde dans la marge, on a donc ajouté la commande \latexcode{\\linebreak} pour inciter \LaTeX à passer à la ligne.}

\begin{latexsource}
\schooldepartment{di}
\end{latexsource}

Les départements actuellement reconnus sont :
\begin{itemize}
  \item \latexcode{di} : Département Informatique
\end{itemize}

\section{Le type de projet}

Le type de projet est défini par la commande \latexcode{\\typereport\{type\}}.
\begin{latexsource}
\typereport{prd}
\end{latexsource}

Les types de projet actuellement reconnus sont :
\begin{itemize}
  \item \latexcode{prd} : Projet Recherche \& Développement
\end{itemize}

\section{L'année du projet}

L'année du projet est définie par la commande \latexcode{\\reportyear\{date\}}. Sauf consigne contraire, la date doit être au format \latexcode{année-année+1}.
\begin{latexsource}
\reportyear{2015-2016}
\end{latexsource}

\section{Le titre}

La commande \latexcode{\\title\{titre\}} permet de spécifier le titre. Pour une mise en page correcte, votre titre ne doit pas dépasser deux lignes sur la page. Si vous souhaitez définir un sous titre, vous pouvez utiliser la commande \latexcode{\\subtitle\{sous titre\}}. Pour une mise en page correcte, votre sous titre ne doit pas dépasser deux lignes sur la page.

\begin{latexsource}
\title{Exemple d'utilisation de la classe de document polytech}
\subtitle{Il peut y avoir un sous titre mais ce n'est pas obligatoire}
\end{latexsource}

\section{Le logo du rapport}

Vous pouvez définir une image illustrant votre travail sur la première de couverture. Pour cela, il vous suffit d'indiquer le chemin relatif d'une image au format JPEG, PNG ou PDF via la commande \latexcode{\\reportlogo\{chemin\}}. Le chemin est relatif au répertoire où se trouve votre fichier principal \LaTeX. Les sous répertoires s'indiquent via le séparateur \latexcode{/}. L'image indiquée par cette commande sera insérée sous le titre et le sous titre sur une hauteur de 4cm.

\begin{latexsource}
\reportlogo{polytech/modeemploi}
\end{latexsource}

\section{Les intervenants}

\label{sec:intervenants}

\subsection{Les étudiants}

Pour indiquer les étudiants rédigeant le rapport, on utilise la commande \latexcode{\\student}. La commande possède trois arguments obligatoires et un argument optionel. L'argument optionel permet d'indiquer la promo de l'étudiant. Les trois paramètres obligatoires indiquent dans l'ordre le prénom, le nom et l'adresse mail de l'étudiant\footnote{On prendra soin d'utiliser les majuscules correctement en début de mot.}. La commande  \latexcode{\\student} peut être utilisée autant de fois que nécessaire.

\begin{latexsource}
\student[DI5]{Pierre}{Premier}{pierre.premier@univ-tours.fr}
\student[DI4]{Paul}{Deuxième}{paul2@univ-tours.fr}
\student{Jacques}{Un peu plus Long}{j.unpeupluslong@example.com}
\end{latexsource}

\subsection{Les tuteurs pédagogique}

Pour indiquer les tuteurs pédagogiques (enseignants, encadrants\ldots), on utilise la commande\linebreak \latexcode{\\academicsupervisor}. La commande possède trois arguments obligatoires et un argument optionel. L'argument optionel permet d'indiquer le département de rattachement du tuteur pédagogique. Le département de rattachement des tuteurs ne doit être indiqué que si les tuteurs n'ont pas tous le même ou si la commande \latexcode{\\schooldepartment} en spécifie un différent. Les trois paramètres obligatoires indiquent dans l'ordre le prénom, le nom et l'adresse mail. La commande \linebreak \latexcode{\\academicsupervisor} peut être utilisée autant de fois que nécessaire.

\begin{latexsource}
\academicsupervisor[Département infomatique]{Jean Noël}{Dupond}{dupond@univ-tours.fr}
\academicsupervisor{Machin}{Chose}{machin.chose@univ-tours.fr}
\end{latexsource}

\subsection{Les tuteurs entreprise}

Pour indiquer les tuteurs entreprise, on utilise la commande\linebreak \latexcode{\\companysupervisor}. La commande possède trois arguments obligatoires et un argument optionel. L'argument optionel permet d'indiquer la fonction du tuteur. La fonction du tuteur ne doit être précisé que si celle-ci est pertinente et significative. Les trois paramètres obligatoires indiquent dans l'ordre le prénom, le nom et l'adresse mail. La commande  \latexcode{\\companysupervisor} peut être utilisée autant de fois que nécessaire.

\begin{center}
\bf Attention : les tuteurs entreprise ne s'afficheront que si l'entreprise est définie avec la commande \latexcode{\\company}.
\end{center}

\begin{latexsource}
\companysupervisor[Fonction très importante]{Un}{Gars}{un.gars@example.com}
\companysupervisor{Un deuxième}{Gars}{undeuxieme.gars@example.com}
\end{latexsource}

\subsection{L'entreprise}

Pour indiquer l'entreprise, on utilise la commande \latexcode{\\company}. La commande possède trois arguments obligatoires et un argument optionel. L'argument optionel permet d'indiquer un logo. Celui-ci doit être au format JPEG, PNG ou PDF et sera dessiné sur une hauteur de 1cm. Les trois paramètres obligatoires indiquent dans l'ordre le nom, l'adresse et l'URL du site web de l'entreprise. La commande  \latexcode{\\company} ne peut être utilisée qu'une seule fois.

\begin{latexsource}
\company[polytech/polytech]{Laboratoire Informatique}{64 avenue Jean Portalis\\37200 Tours}{li.univ-tours.fr}
\company{Laboratoire Informatique}{64 avenue Jean Portalis, 37200 Tours}{li.univ-tours.fr}
\end{latexsource}

\section{Les mots clés}

Les mots clés en français et en anglais sont spécifiés par les commandes \latexcode{\\motcle} et \latexcode{\\keyword} utilisées autant de fois que nécessaire. D'une manière générale, vous devez choisir soigneusement vos mots clés pour refléter le contenu du rapport. Les versions française ou anglaise des mots clés ne sont pas forcément une traduction de l'un en l'autre et n'apparaissent pas forcément en même nombre.

\begin{latexsource}
\motcle{mot}
\motcle{clé}
\motcle{deux mots}
\keyword{word}
\keyword{key}
\keyword{two words}
\keyword{fourth word}
\end{latexsource}

\section{Les résumés}

Les résumés français et anglais sont spécifiés par les commandes \latexcode{\\resume} et \latexcode{\\abstract}. Vous pouvez avoir des paragraphes au sein de vos résumés en laissant une ligne vide. Dans tous les cas, vous devez dimensionner vos résumés de manière à ce que la dernière de couverture tienne sur une seule page. Vous devez apporter le même soin à la rédaction de vos résumés qu'au choix de vos mots clés : c'est la première image global de votre rapport. Ils doivent donc en résumer l'essentiel.

\begin{latexsource}
\resume{ Lorem ipsum dolor sit amet, consectetur adipiscing elit. Vestibulum dapibus
sit amet sapien sed sodales. Pellentesque faucibus lacus mauris, id semper odio viverra
eu. Aliquam erat volutpat. Proin ultricies velit et tortor consequat cursus. Curabitur
ac elit suscipit ligula iaculis auctor. Sed tristique nulla libero, non tempor nibh
laoreet vel. Cras id congue diam. Vestibulum tellus felis, egestas non diam a,
rhoncus eleifend metus. Aliquam ligula mauris, pharetra sit amet erat a, luctus
rhoncus arcu. Vivamus ac metus mauris.

Nam mattis elit eget lectus rhoncus euismod. Etiam fermentum diam velit, ac dapibus
ante malesuada eget. Phasellus mauris augue, convallis eu nulla ac, ornare luctus
augue. Praesent fringilla urna sed lacus viverra placerat. Nunc pulvinar facilisis
ultrices. Nulla neque justo, suscipit sit amet dignissim at, facilisis eget metus.
Curabitur venenatis, ante bibendum volutpat accumsan, eros diam faucibus neque, at
mattis augue massa ut nisl. Integer enim leo, sodales at euismod vel, efficitur vel
magna. Etiam congue, augue in aliquet rutrum, magna elit sodales metus, quis
hendrerit purus urna hendrerit purus. Quisque euismod leo at felis efficitur, vel
posuere tortor accumsan. Pellentesque habitant morbi tristique senectus et netus et
malesuada fames ac turpis egestas. Quisque eget vestibulum augue.}
\abstract{ Lorem ipsum dolor sit amet, consectetur adipiscing elit. Vestibulum
dapibus sit amet sapien sed sodales. Pellentesque faucibus lacus mauris, id semper
odio viverra eu. Aliquam erat volutpat. Proin ultricies velit et tortor consequat
cursus. Curabitur ac elit suscipit ligula iaculis auctor. Sed tristique nulla libero,
non tempor nibh laoreet vel. Aliquam ligula mauris, pharetra sit amet erat a, luctus
rhoncus arcu. Vivamus ac metus mauris.

Nam mattis elit eget lectus rhoncus euismod. Etiam fermentum diam velit, ac dapibus
ante malesuada eget. Phasellus mauris augue, convallis eu nulla ac, ornare luctus
augue. Praesent fringilla urna sed lacus viverra placerat. Nunc pulvinar facilisis
ultrices. Nulla neque justo, suscipit sit amet dignissim at, facilisis eget metus.
Sed rhoncus in dolor a facilisis. Mauris in consectetur nisl. Curabitur venenatis,
ante bibendum volutpat accumsan, eros diam faucibus neque, at mattis augue massa
ut nisl. Integer enim leo, sodales at euismod vel, efficitur vel magna. Etiam
congue, augue in aliquet rutrum, magna elit sodales metus, quis hendrerit purus
urna hendrerit purus. Quisque euismod leo at felis efficitur, vel posuere tortor
accumsan. Pellentesque habitant morbi tristique senectus et netus et malesuada
fames ac turpis egestas. Quisque eget vestibulum augue.}
\end{latexsource}

\section{Structure minimale de votre document}

\begin{latexsource}
\documentclass[overfullbox]{polytech}

% ajout d'éventuels packages et commandes
% \usepackage{...}

% renseignement de la première et dernière de couverture (\student...)

\begin{document}
\maketitle

% contenu du rapport

\makelastpage
\end{document}
\end{latexsource}

\chapter{Tables et figures}

\label{chap:tablesetfigures}

Pour insérer des tables et des figures, vous ne devez pas utiliser les environnements standards \latexcode{table} et \latexcode{figure}. en lieu et place, vous devez utiliser les environnements \latexcode{Table} et \latexcode{Figure}. Ces environnements admettent deux paramètres : l'identifiant de référence (label) et le titre de la table/figure. Le positionnement dans le document se fait de façon automatique en haut de page. Vous ne pouvez pas modifier ce placement automatique. Il est donc très important que l'ensemble de vos tables et figures soient référencées dans le texte (cf. \autoref{chap:ref}).

\begin{latexsource}
Pendant que j'y pense, j'ai une belle image qui illustre ma pensée
(cf. \autoref{fig:modeemploi}) que je peux compléter par la \autoref{tab:deficits}

\begin{Figure}{fig:modeemploi}{Ma superbe image}
	\pgfimage{polytech/modeemploi}
\end{Figure}

\begin{Table}{tab:deficits}{Déficit public de la France : on est mal barré !}
	\begin{tabu}{cccccccc}
			&2007	&2008	&2009	&2010	&2011	&2012	&2013\\\hline
		Milliards \euro&1216	&1324	&1499	&1602	&1725	&1841	&1925\\\hline
		\% PIB	&64.4\%	&68.5\%	&79.5\%	&82.7\%	&86.2\%	&90.6\%	&93.5\%\\\hline
	\end{tabu}
\end{Table}
\end{latexsource}

Ce qui donne :

Pendant que j'y pense, j'ai une belle image qui illustre ma pensée
(cf. \autoref{fig:modeemploi}) que je peux compléter par la \autoref{tab:deficits}

\begin{Figure}{fig:modeemploi}{Ma superbe image}
	\pgfimage{polytech/modeemploi}
\end{Figure}

\begin{Table}{tab:deficits}{Déficit public de la France : on est mal barré !}
	\begin{tabu}{cccccccc}
			&2007	&2008	&2009	&2010	&2011	&2012	&2013\\\hline
		Milliards \euro&1216	&1324	&1499	&1602	&1725	&1841	&1925\\\hline
		\% PIB	&64.4\%	&68.5\%	&79.5\%	&82.7\%	&86.2\%	&90.6\%	&93.5\%\\\hline
	\end{tabu}
\end{Table}


\chapter{Références}

\label{chap:ref}

\section{Nommer des emplacements avec \texorpdfstring{\latexcode{\\label}}{\textbackslash{}label}}

Comme vous le savez, \LaTeX est particulièrement pratique lorsqu'il s'agit de définir des références croisées ou non dans un document. Pour nommer un emplacement, il suffit d'utiliser la commande \latexcode{\\label} en lui donnant en paramètre un nom symbolique représentant l'emplacement. Tous les caractères ne sont pas utilisables pour le nom (cf. référence \LaTeX). Classiquement, on utilise les lettres non accentuées (a-z, A-Z), les chiffres (0-9) et les symboles \latexcode{.-+_:}. Lorsque le nombre d'emplacement à utiliser devient important, il est nécessaire de s'organiser. Pour cela, vous pouvez adopter les règles simples suivantes :
\begin{itemize}
  \item Le nom de l'emplacement doit être sous la forme \latexcode{type:nom}. \latexcode{type} représente le type d'emplacement tel que \latexcode{part}, \latexcode{chap}, \latexcode{sec}, \latexcode{eq}\ldots
  \item Le nom doit être descriptif de l'emplacement. Ainsi, \latexcode{chap1} et \latexcode{sec2.1} doivent absolument être évités et remplacés par \latexcode{chap:trousnoirs} et \latexcode{sec:bigbang_theory}.
  \item
\end{itemize}

\section{Référencer un emplacement}
\label{sec:ref}

Pour référencer un emplacement, plusieurs commandes sont à votre disposition. Les commandes les plus utiles sont :
\begin{itemize}
  \item \latexcode{\\ref} : insère le numéro identifiant l'emplacement (partie, chapitre, section, équation\ldots).
  \item \latexcode{\\autoref} (recommandé) : identique à \latexcode{\\ref} mais préfixe le numéro du type de référence.
  \item \latexcode{\\eqref} : insère le numéro de l'équation entre parenthèses
  \item \latexcode{\\pageref} : insère le numéro de page
  \item \latexcode{\\refchapterof} : insère le type de chapitre/annexe suivi de numéro correspondant au label donné
\end{itemize}

Dans les styles du document, le numéro de chapitre a été retiré des différents identifiants. Ainsi un chapitre peut avoir la Figure 1 et un autre la Figure 1. Les numérotations sont donc locales au chapitre. Lorsque l'on référence un emplacement, si cet emplacement est dans le même chapitre alors le numéro du chapitre est omis. Dans le cas contraire, le numéro du chapitre est automatiquement ajouté.

Pour mieux comprendre l'utilisation et l'effet de ces commandes, voici quelques exemples.

%  label, remarque
\newcommand{\showreferences}[2]{
	\begin{tabu}{|c|X[c]|c|X[c]|}
		\hline
		\multicolumn{2}{|c}{\latexcodeintab!\\label\{#1\}!}&\multicolumn{2}{c|}{#2}\\\hline
		\latexcodeintab!\\refchapterof!& \refchapterof{#1}&
		\latexcodeintab!\\pageref! & \pageref{#1}\\\hline
		\latexcodeintab!\\ref! & \ref{#1} & \latexcodeintab!\\autoref! & \autoref{#1}\\\hline
	  \latexcodeintab!\\eqref! & \multicolumn{3}{c|}{\eqref{#1}}\\
	  \hline
	\end{tabu}\par
}

\showreferences{inconnue}{référence inconnue}
\showreferences{chap:ref}{ce chapitre}
\showreferences{sec:ref}{section de ce chapitre}
\showreferences{eq:energie}{équation de ce chapitre}
\showreferences{fig:modeemploi2}{figure de ce chapitre}
\showreferences{chap:tablesetfigures}{un autre chapitre}
\showreferences{sec:intervenants}{section d'un autre chapitre}
\showreferences{eq:energie2}{équation d'un autre chapitre}
\showreferences{fig:modeemploi}{figure d'un autre chapitre}
\showreferences{chap:ref3}{une annexe}
\showreferences{sec:sertarien}{section d'une annexe}
\showreferences{eq:energie3}{équation d'une annexe}
\showreferences{fig:modeemploi3}{figure d'une annexe}

\vspace{1cm}
\par
\hrule\vspace{0.1cm}\hrule
\vspace{1cm}

La suite de ce chapitre ne contient pas d'instructions utiles mais uniquement des exemples permettant d'illustrer ce chapitre. Vous pouvez donc passer directement à la suite.

\begin{equation}
E=mc^2
\label{eq:energie}
\end{equation}

\begin{Figure}{fig:modeemploi2}{Ma superbe image : une deuxième fois car je l'aime bien\ldots}
	\pgfimage{polytech/modeemploi}
\end{Figure}


\chapter{Références \textit{I'll be back !}}

\label{chap:ref2}

Ce chapitre ne contient pas d'instructions utiles mais uniquement des exemples permettant d'illustrer le chapitre précédent. Vous pouvez donc passer directement au chapitre suivant.

\begin{equation}
E=mc^2
\label{eq:energie2}
\end{equation}


\begin{Figure}{fig:modeemploix}{Ma superbe image encore !}
	\pgfimage{polytech/modeemploi}
\end{Figure}

\begin{Figure}{fig:modeemploiy}{Ma superbe image encore une fois !}
	\pgfimage{polytech/modeemploi}
\end{Figure}

\chapter{Insérer du code}

\label{chap:codesource}

Pour insérer du code dans votre document, vous avez à disposition le package \latexcode{listings}. Ce package est très riche en possibilités, il vous est fortement recommandé de consulter sa documentation si vous souhaitez l'exploiter en profondeur.

\section{Commandes simplifiées}

Pour simplifier son utilisation, plusieurs commandes et environnements supplémentaires ont été défini. Pour chacun des langages supportés, trois commandes et un environnements sont disponibles. En notant \latexcode{XXX} le nom du langage, on a :
\begin{itemize}
  \item La commande \latexcode{\\XXXcode} :\par
  	\begin{itemize}
		  \item Commande identique à la commande \latexcode{\\lstinline\{ \}} mais pré-configurée pour le langage \latexcode{XXX}.
		  \item Son argument est le code source à afficher.
		  \item Elle s'utilise dans le corps d'un texte (ex: un phrase).
		  \item Les caractères spéciaux tels que \textbackslash, \{ et \} doivent être préfixés d'un \textbackslash.
		  \item \textbf{Cette commande est fragile et ne fonctionne pas dans les tableaux.} Il faut utiliser le commande \latexcode{\\XXXcodeintab} dans ce cas là.
		  \item Example : \latexcode{\\latexcode\{\\color\{red\}mise en couleur rouge\}}
		\end{itemize}
  \item La commande \latexcode{\\XXXcodeintab} :\par
   	\begin{itemize}
		  \item Commande identique à la commande \latexcode{\\lstinline! !} mais pré-configurée pour le langage \latexcode{XXX}.
		  \item Les parenthèses sont remplacés par un caractère de début et un caractère de fin (choix arbitraire mais identique).
		  \item Tout ce qui est entre ces deux caractères est le code source à afficher.
		  \item Elle s'utilise dans le corps d'un texte (ex: un phrase).
		  \item Les caractères spéciaux tels que \textbackslash, \{ et \} doivent être préfixés d'un \textbackslash.
		  \item Cette commande fonctionne également dans les tableaux.
		  \item Example : \latexcode{\\latexcodeintab!\\color\{red\}mise en couleur rouge!}\\ ou \latexcode{\\latexcodeintab+\\color\{red\}mise en couleur rouge+}
		\end{itemize}
	\item La commande \latexcode{\\XXXsourcefile} :\par
		\begin{itemize}
		  \item Commande identique à \latexcode{\\lstinputlisting} mais pré-configurée pour le langage \latexcode{XXX}.
		  \item Le paramètre est le chemin relatif du fichier dont le code doit être affiché.
		  \item Exemple : \latexcode{\\latexsourcefile\{main.tex\}}
		\end{itemize}
	\item L'environnement \latexcode{\\begin\{XXXsource\} \\end\{XXXsource\}} :
		\begin{itemize}
		  \item Environnement identique à \latexcode{\\begin\{lstlisting\} \\end\{lstlisting\}} mais pré-configurée pour le langage \latexcode{XXX}.
		  \item Tout ce qui est défini entre le début et la fin de l'environnement est le code à afficher.
		  \item Par défaut, les lignes sont numérotés
		  \item Exemple :
\begin{lstlisting}[style=polytech@lst@base,language=TeX]
\begin{latexsource}
\textbf{text mis en gras}
\end{latexsource}
\end{lstlisting}
		\end{itemize}
\end{itemize}

\section{Paramètres supplémentaires}

Toutes ces commandes et environnements admettent un paramètre optionnel permettant de changer leur comportement. Pour le détail de ce que vous pouvez faire, consultez la documentation de \href{https://www.ctan.org/pkg/listings}{listings} sur le CTAN.

\begin{lstlisting}[style=polytech@lst@base,language=TeX]
\latexcode[basicstyle=\color{red}]{\\color\{red\}mise en couleur rouge}
\latexcodeintab[basicstyle=\color{red}]{\\color\{red\}mise en couleur rouge}
\latexsourcefile[basicstyle=\color{red}]{main.tex}
\begin{latexsource}[basicstyle=\color{red}]
\textbf{text mis en gras}
\end{latexsource}
\end{lstlisting}

\section{Styles prédéfinies}

Si pour une raison ou une autre vous étiez obligé de revenir à l'utilisation des commandes originales de listings, vous pouvez obtenir le même aspect du résultat en spécifiant l'option :
\begin{itemize}
  \item \latexcode{style=polytech@lst@inline} pour le code \textit{inline} (\latexcode{\\lstinline}) et
  \item \latexcode{style=polytech@lst@base} pour les autres.
\end{itemize}


\section{Langages supportés}

Les langages pour lesquels les commandes simplifiées ont été définis sont donnés par la \autoref{tab:lstlang}.

\begin{Table}{tab:lstlang}{Liste des langages avec commandes simplifiés pour le code source}
\begin{tabu}{|c|c|}
\hline
\bf Langage&\bf Préfix des commandes\\\hline
C&c\\\hline
C++&cpp\\\hline
Java&java\\\hline
Bash&bash\\\hline
Make&make\\\hline
Javascript&javascript\\\hline
XML&xml\\\hline
SQL&sql\\\hline
PHP&php\\\hline
Python&python\\\hline
HTML&html\\\hline
Scilab&scilab\\\hline
Matlab&matlab\\\hline
\LaTeX&latex\\\hline
\end{tabu}
\end{Table}

Le package listings supporte de nombreux langages et vous pouvez même définir les règles de mise en forme pour votre propre langage (cf. manuel de listings).
Pour créer les commandes simplifiées pour un langage, il suffit d'utiliser la commande \latexcode{\\newsourceformat\{cpp\}\{C++\}}. Dans cet exemple, \latexcode{cpp} est le préfixe utilisé pour les commandes et \latexcode{C++} est le nom du langage tel qu'il est connu par la package listings.


\chapter{Rapports hebdomadaires}

Pour votre projet, vous devez inclure en fin de document, avant la commande \latexcode{\\makelastpage}, vos comptes rendus hebdomadaires.
Chaque compte rendus est défini avec la commande \latexcode{\\weeklyreport} en indiquant la date puis le texte du compte rendu. Vous pouvez y inclure quasiment n'importe quel code \LaTeX.

\begin{latexsource}
\weeklyreport{XX/XX/XXXX}{
	\lipsum[1-2]
}

\weeklyreport{XX/XX/YYYY}{
	\lipsum[3]
}
\end{latexsource}


\chapter{Bibliographie}

La bibliographie est géré via bibtex. Pour inclure un fichier de bibliographie, il vous suffit d'utiliser la commande
\latexcode{\\bibliography\{biblio\}} avant la commande \latexcode{\\makelastpages}.


Si vous renseignez le champs bibtex \latexcode{annote} alors la clé de citation et le contenu de ce champs apparaîtront dans la bibliographie.

Pour gérer votre bibliographie, vous pouvez utiliser n'importe quel logiciel de gestion de bibliographie bibtex tel que JabRef.
Sous JabRef, pour avoir accès au champs \latexcode{annote} lors de l'édition d'une entrée bibliographique, il vous suffit d'aller dans le menu \textit{Options}, sous menu \textit{Set up general fields} et d'ajouter la ligne \latexcode{Annote:annote}.

\chapter{Gestion des overfull box}

Lorsque vous avez un débordement de boite (typiquement dans la marge), l'option du document \latexcode{overfullbox} vous aide à les repérer en les affichant. Pour les éliminer, il existe de nombreux moyens. en voici quelques exemples :
\begin{itemize}
  \item \LaTeX n'arrive pas à couper le mot \latexcode{abcdefghi} pour en passer une partie à la ligne : il vous suffit d'ajouter dans le préambule la commande \latexcode{\\hyphenation\{abcd-ef-ghi\}} pour informer \LaTeX que le mot peut être coupé après le 4\ieme et le 6\ieme caractères. L'information est applicable à tout le document. Une alternative ponctuelle consiste à utiliser \latexcode{abcd\\-ef\\-ghi} directement dans le texte.
  \item \LaTeX n'arrive pas à couper un mot et  \latexcode{\\hyphenation} n'est pas utilisable. Il suffit d'ajouter la commande \latexcode{\\linebreak} pour indiquer à \LaTeX qu'il \textbf{serait} bien de passer à la ligne à cet endroit.
  \item Pour indiquer à \LaTeX qu'il \textbf{serait} bien de passer à la page suivante, vous pouvez utiliser \latexcode{\\pagebeak}. La commande \latexcode{\\newpage} impose un changement de page inconditionnel.
\end{itemize}

\chapter{Les packages}

La classe de document importe automatiquement plusieurs packages que vous pouvez donc utiliser dans votre document :
\begin{itemize}
  \item \latexcode{url} : gestion des urls avec \latexcode{\\url}
  \item \latexcode{amsmath}, \latexcode{amsfonts} : plein de commandes et de symboles pour les maths
  \item \latexcode{kpfonts}, \latexcode{marvosym}, \latexcode{stmaryrd} : pleins de caractères en plus
  \item \latexcode{listings} : mise en forme de code source (cf. \autoref{chap:codesource})
  \item \latexcode{tabu} : une alternative à \latexcode{tabular} plus flexible
  \item \latexcode{lscape} : mode paysage
  \item \latexcode{comment} : pour commenter de grosse portion de code via l'environnement \latexcode{comment}
  \item \latexcode{pgf}, \latexcode{tikz} : pour gérer les images (au format JPEG, PNG, PDF) et faire des dessins
  \item \latexcode{algorithm2e} : pour écrire des algorithmes
  \item \latexcode{eurosym} : pour le symbole \euro
  \item \latexcode{hyperref} : pour faire des liens
\end{itemize}

\appendix
\chapter{Références \textit{I'll be back ! Again !}}

\label{chap:ref3}

Cette annexe ne contient pas d'instructions utiles mais uniquement des exemples permettant d'illustrer le chapitre sur les références. Vous pouvez donc passer directement à la suite.

\section{Une section qui ne sert à rien}
\label{sec:sertarien}

\begin{equation}
E=mc^2
\label{eq:energie3}
\end{equation}

\begin{Figure}{fig:modeemploi3}{Ma superbe image une troisième fois !}
	\pgfimage{polytech/modeemploi}
\end{Figure}

\end{document}