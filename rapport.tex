\documentclass[overfullbox]{polytech/polytech}

% LISEZ LE MODE D'EMPLOI POUR L'INTEGRALITE DES CONSIGNES

% on ajoute ici des packages supplémentaires
% Attention : il peut y avoir des incompatibilités avec la classe de document

\usepackage{lipsum} %permet de générer du texte

% on indique le département concerné (di)
\schooldepartment{di}
% on indique le type de projet concerné (prd)
\typereport{prd}
% on indique l'année en cours
\reportyear{2015-2016}

% on donne un titre au travail (sans dépasser 2 lignes à l'affichage)
\title{Application d'aide à l'interaction homme/machine pour les personnes handicapées}
% on peut donner un sous titre (sans dépasser 2 lignes à l'affichage) mais ce n'est pas nécessaire
%\subtitle{Il peut y avoir un sous titre mais ce n'est pas obligatoire}
% on peut donner un logo illustrant le projet (hauteur d'affichage 4cm) mais ce n'est pas nécessaire
%\reportlogo{polytech/modeemploi}

% on indique les contributeurs à ce rapport

% Le(s) étudiant(s) aux formats suivants :
% \student{Prénom}{Nom}{Mail}
% \student[Année d'études]{Prénom}{Nom}{Mail}
\student[DI5]{Florian}{Tissier}{florian.tissier@etu.univ-tours.fr}
% Le(s) superviseur(s) académique (ou encadrant(s)) aux format suivants :
% \academicsupervisor{Prénom}{Nom}{mail}
% \academicsupervisor[Affiliation]{Prénom}{Nom}{mail}
\academicsupervisor[Département infomatique]{Mohamed}{Slimane}{mohamed.slimane@univ-tours.fr}
\academicsupervisor[Département infomatique]{Donatello}{Conte}{donetello.conte@univ-tours.fr}
% Le(s) tuteur(s) entreprise aux formats suivants :
% \companysupervisor{Prénom}{Nom}{Mail}
% \companysupervisor[Fonction]{Prénom}{Nom}{Mail}
% L'entreprise aux formats suivants :
%	\company{Nom de l'entreprise}{Adresse}{URL du site web}
% \company[logo entreprise]{Nom de l'entreprise}{Adresse}{URL du site web}
% S'il est indiqué le logo de l'entreprise s'affichera sur une hauteur de 1cm
% Attention : pour que les tuteurs entreprise s'affichent, l'entreprise doit être définie
%\company[polytech/polytech]{Laboratoire Informatique}{64 avenue Jean Portalis\\37200 Tours}{http://li.univ-tours.fr}
%\company{Laboratoire Informatique}{64 avenue Jean Portalis, 37200 Tours}{li.univ-tours.fr}


% On indique les mots clés avec \motcle{mot clé} en français et \keyword{keyword} en anglais
% Le résumé significatif et descriptif du contenu du rapport en 5 à 10 lignes se spécifie par \resume{...} en français et \abstract{...} en anglais
% Attention : tout doit tenir sur la dernière page
\resume{ Lorem ipsum dolor sit amet, consectetur adipiscing elit. Vestibulum dapibus sit amet sapien sed sodales. Pellentesque faucibus lacus mauris, id semper odio viverra eu. Aliquam erat volutpat. Proin ultricies velit et tortor consequat cursus. Curabitur ac elit suscipit ligula iaculis auctor. Sed tristique nulla libero, non tempor nibh laoreet vel. Cras id congue diam. Vestibulum tellus felis, egestas non diam a, rhoncus eleifend metus. Aliquam ligula mauris, pharetra sit amet erat a, luctus rhoncus arcu. Vivamus ac metus mauris.

Nam mattis elit eget lectus rhoncus euismod. Etiam fermentum diam velit, ac dapibus ante malesuada eget. Phasellus mauris augue, convallis eu nulla ac, ornare luctus augue. Praesent fringilla urna sed lacus viverra placerat. Nunc pulvinar facilisis ultrices. Nulla neque justo, suscipit sit amet dignissim at, facilisis eget metus. Curabitur venenatis, ante bibendum volutpat accumsan, eros diam faucibus neque, at mattis augue massa ut nisl. Integer enim leo, sodales at euismod vel, efficitur vel magna. Etiam congue, augue in aliquet rutrum, magna elit sodales metus, quis hendrerit purus urna hendrerit purus. Quisque euismod leo at felis efficitur, vel posuere tortor accumsan. Pellentesque habitant morbi tristique senectus et netus et malesuada fames ac turpis egestas. Quisque eget vestibulum augue.}
% chaque mot clé ou groupe de mots clés est défini via la commande \motcle en français
\motcle{mot}
\motcle{clé}
\motcle{deux mots}

% résumé en anglais
\abstract{ Lorem ipsum dolor sit amet, consectetur adipiscing elit. Vestibulum dapibus sit amet sapien sed sodales. Pellentesque faucibus lacus mauris, id semper odio viverra eu. Aliquam erat volutpat. Proin ultricies velit et tortor consequat cursus. Curabitur ac elit suscipit ligula iaculis auctor. Sed tristique nulla libero, non tempor nibh laoreet vel. Aliquam ligula mauris, pharetra sit amet erat a, luctus rhoncus arcu. Vivamus ac metus mauris.

Nam mattis elit eget lectus rhoncus euismod. Etiam fermentum diam velit, ac dapibus ante malesuada eget. Phasellus mauris augue, convallis eu nulla ac, ornare luctus augue. Praesent fringilla urna sed lacus viverra placerat. Nunc pulvinar facilisis ultrices. Nulla neque justo, suscipit sit amet dignissim at, facilisis eget metus. Sed rhoncus in dolor a facilisis. Mauris in consectetur nisl. Curabitur venenatis, ante bibendum volutpat accumsan, eros diam faucibus neque, at mattis augue massa ut nisl. Integer enim leo, sodales at euismod vel, efficitur vel magna. Etiam congue, augue in aliquet rutrum, magna elit sodales metus, quis hendrerit purus urna hendrerit purus. Quisque euismod leo at felis efficitur, vel posuere tortor accumsan. Pellentesque habitant morbi tristique senectus et netus et malesuada fames ac turpis egestas. Quisque eget vestibulum augue.}

% chaque mot clé ou groupe de mots clés est défini via la commande \motcle en anglais
% Attention : il n'y a pas forcément une traduction directe entre les mots clés et les keywords
\keyword{word}
\keyword{key}
\keyword{two words}
\keyword{fourth word}



\bibliography{biblio}
% le document commence ici
\begin{document}

% on commence par générer la page de titre, la liste des intervenants, les tables des matières, figures, tables, listings
\maketitle


% Le chapitre d'introduction générale n'est souvent pas numéroté
% La commande \unnumberedchapter fonctionne de façon identique à \chapter mais produit un chapitre non numéroté présent dans la table des matières (à la différence de \chapter*)
% Attention : un chapitre sans numéro (et ses sections) ne peu(ven)t pas être référencé(s) dans le document (\label, \ref)
\unnumberedchapter[Introduction]{Introduction et présentation du projet}

Monsieur Slimane étant membre d'une association s'occupant de personnes handicapées, il souhaitait pouvoir aider et faciliter la vie de ces derniers via un projet réalisé au sein de l'école.\\
Dans ce projet de recherche et développement, mon but est de construire un système permettant, à partir d'un flux vidéo, de détecter l'émotion actuelle d'une personne handicapée dans le but de réaliser certaines actions pouvant améliorer son bien être.\\
\\
Ce rapport va donc se diviser en deux grandes parties: tout d'abord la partie Recherche qui va contenir toutes les recherches que j'ai pu réalisées ainsi que toutes les bases théoriques dont j'aurais besoin par la suite. La deuxième partie est la partie Développement qui va se concentrer sur les différentes étapes du développement de cette application d'aide aux personnes handicapées.\\
\\
Dans la partie Recherche de ce rapport, je vais tout d'abord vous présenter les deux façons majeures permettant de définir les mouvements du visages qui composent une émotion.\\
Je vous présenterai ensuite les différentes techniques permettant de capturer un visage, que ce soit en 2D ou en 3D, ainsi que les avantages et inconvénient de chacune de ces techniques.\\
Je continuerai ensuite en présentant les différentes étapes nécessaires à la construction d'un système de reconnaissance faciale d'émotions performant et efficace. Je décrirai également l'état de l'art de chacune de ces étapes.\\
Je ferai ensuite une présentation des bases de données publiques les plus connues utilisées pour la reconnaissance d'émotions.\\
Je finirai ensuite par définir concrètement de quoi se compose chaque partie de l'application que j'ai développé, ainsi que les choix qui ont permit cette définition.\\
\\
Dans la partie Développement, ....


% Je peux diviser le rapport en partie
\part{Recherche}
\label{part:part_recherche}

\chapter{Le tout premier truc}
\label{chap:toutpremier}

\lipsum[1]

\section{Pour ne pas se casser la tête}

\label{sec:cassetete}

Pour ne pas se casser la tête à rédiger mon rapport, je lis le mode d'emploi (cf. \ref{fig:modeemploi})
Toute figure ou table doit avoir un titre et être référencée dans le texte (sinon ça ne sert à rien).
Pour cela, vous devez utiliser deux environnements \texttt{Figure} et \texttt{Table}. Chacun de ces environments admet deux paramètres : le label et le titre de la figure/table. Le positionnement de la figure/table étant contrôlé par la classe de document, il est très important que ces figures/tables soient référencées dans le texte (ex: \autoref{fig:modeemploi}, \autoref{fig:modeemploi2}, \autoref{tab:deficits}).

\begin{Figure}{fig:modeemploi}{Ma superbe image}
\pgfimage{polytech/modeemploi}
\end{Figure}

\begin{Table}{tab:deficits}{Déficit public de la France : on est mal barré !}
\begin{tabu}{cccccccc}
	&2007	&2008	&2009	&2010	&2011	&2012	&2013\\\hline
Milliards \euro&1216	&1324	&1499	&1602	&1725	&1841	&1925\\\hline
\% PIB	&64.4\%	&68.5\%	&79.5\%	&82.7\%	&86.2\%	&90.6\%	&93.5\%\\\hline
\end{tabu}
\end{Table}

\begin{Figure}{fig:modeemploi2}{Ma superbe image encore !!}
\pgfimage{polytech/modeemploi}
\end{Figure}

\part{essai}
\label{part:essai}

\chapter{truc}
\lipsum[1-20]

\section{Ma section}
\lipsum[1-5]

\section{Ma section}
\lipsum[1-5]

\subsection{Ma section}
\lipsum[1-5]
\subsection{Ma section}
\lipsum[1-5]
\subsubsection{Ma section}
\lipsum[1-5]
\subsubsection{Ma section}
\lipsum[1-5]
\paragraph{Ma section}
\lipsum[1-5]
\paragraph{Ma section}
\lipsum[1-5]
\subparagraph{Ma section}
\lipsum[1-5]
\subparagraph{Ma section}
\lipsum[1-5]

\unnumberedchapter{Conclusion}

\appendix

\chapter{Ma première annexe}

\lipsum[1-4]

\chapter{Ma deuxième annexe}

\cite{article}

\lipsum[1-4]

\weeklyreport{17/09/2015}{
	Découverte de 2 grandes méthodes de description des mouvements du visage existent : le FACS (Facial Action Coding System) mis en place par P. Ekman et w. Friesen en 1978 et le FAPU (Facial Animation Parameter Units) introduit par la norme de codage vidéo MPEG-4.\\
	Recherche sur les types d'acquisitions d'images en 2D ou en 3D avec le matériel nécessaires à chaque fois ainsi que les algorithmes disponibles.
}
\weeklyreport{24/09/2015}{
	Recherche sur comment se décompose un bon système de reconnaissance facial d'émotions.\\
	Décomposition en 4 parties (récupération du visage, normalisation, extraction des points clés, classification) et recherche plus poussée sur les 2 premières parties.
}
\weeklyreport{01/10/2015}{
	Continuation des recherches sur les 2 dernières parties du système.\\
	Recherche également sur les différentes bases de données 2D publics disponibles à l'utilisation.
}
\weeklyreport{08/10/2015}{
	Recherche plus approfondies sur les filtres de Gabor. Présentation de mes recherches à Messieurs Conte et Slimane.\\
	Commencement de l'écriture du rapport.
}
\weeklyreport{15/10/2015}{

}
\weeklyreport{22/10/2015}{
	
}
\weeklyreport{05/11/2015}{
	
}
\weeklyreport{12/11/2015}{
	
}
\weeklyreport{19/11/2015}{
	
}
\weeklyreport{26/11/2015}{
	
}

% petite astuce pour tout citer : A NE PAS REPRODUIRE DANS VOTRE RAPPORT
\nocite{*}


\makelastpages

\end{document}


